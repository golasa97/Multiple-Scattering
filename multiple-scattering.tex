The capture correction term is used to correct for both resonance self-shielding, as well as multiple scattering. Multiple scattering is an effect which occurs in which a neutron is scattered one or more times before being captured in a sample. As the thickness of the sample increases, the probability of a scattered neutron experiencing a subsequent reaction (that being a capture or additional scattering events) also increases. Therefore, neutrons which undergo scattering events can end up miscounted as captures, artificially increasing the measured yield.

This correction is made with reference to the thin sample approximation,
\begin{equation}
    \label{eq:thin-sample-approximation}
    \left\langle Y \right\rangle \approx n \left\langle \sigma_\gamma \right\rangle 
\end{equation}
where $\left\langle Y \right\rangle$ is the average yield, $n$ is the sample thickness, and $\left\langle \sigma_\gamma \right\rangle$ is the average capture cross section. The capture correction $C_C$ corrects the thin sample approximation such that
\begin{equation}
    \label{eq:correction-thin-sample-approximation}
    \left\langle Y \right\rangle = n \left\langle \sigma_\gamma \right\rangle C_C
\end{equation}

In order to calculate the correction, a series of Monte Carlo simulations are used to simulate a neutron undergoing a series of capture or scattering events in a cylindrical disc target. The target has a radius $R$, is parallel to the $z$-axis, and with front and rear faces located at $z=0$ and $z=n$ respectively, which are centered at $(x,y)=(0,0)$. A neutron is born at the energy of interest, $E^0$, and is incident on the front face of the target, with the location
\begin{equation}
    \label{eq:incident-neutron-location}
    \overrightarrow{x}^0 = \begin{bmatrix}
        x^0 \\
        y^0 \\
        z^0
    \end{bmatrix} =
    \begin{bmatrix} 
        r_0\cos{\left( \theta_0 \right)} \\
        r_0\sin{\left( \theta_0 \right)} \\
        0
    \end{bmatrix}
\end{equation}
where
\begin{align*}
    r_0 &= R\sqrt{\zeta}, \\
    \theta_0 &= 2\pi\zeta,
\end{align*}
and $\zeta$ is a uniformly distributed random value between 0 and 1. The neutron is also traveling along the $z$-axis, and therefore is born with the direction vector
\begin{equation}
    \label{eq:incident-neutron-location}
    \overrightarrow{\Omega}^0 = \begin{bmatrix}
        \Omega^0_u \\
        \Omega^0_v \\
        \Omega^0_w
    \end{bmatrix} =
    \begin{bmatrix}
        0 \\
        0 \\
        1
    \end{bmatrix}
\end{equation}

Cross sections are sampled at the energy $E^0$ according to the procedure described in \autoref{sec:resonance-sampling}. These values are then used to produce the average cross sections,
\begin{align}
    \label{eq:average-cross-sections}
    \Sigma_{t}^{0} &= \sum_{j} \sigma_{t,j}^{0} \delta_{j} \\
    \Sigma_{\gamma}^{0} &= \sum_{j} \sigma_{\gamma,j}^{0} \delta_{j} \\
    \Sigma_{el}^{0} &= \sum_{j} \sigma_{el,j}^{0} \delta_{j}
\end{align}
where $\Sigma_{t}^0$, $\Sigma_{\gamma}^0$, $\Sigma_{el}^0$ are the macroscopic total, capture, and elastic scattering cross sections respectively, and $\sigma_{j}^0$ and $\delta_{j}$ are the microscopic cross section and relative abundance of the $j^{th}$ isotope, respectively.

\subsection{Scattering Events}

\subsubsection{Sampling Location}
\label{sec:sampling-location-ms}
The distance to leave the sample is then calculated according to the shortest path to intersecting with one of the surfaces in the direction that the neutron is traveling. For the initial neutron, i.e., before any scattering events have occurred, this is just the thickness of the sample $n$. However, for a scattering event this must be solved generally for a neutron that undergone $k$ scattering events. In this case, the neutron would have position vector $\overrightarrow{x}^k$ and direction $\overrightarrow{\Omega}^k$. The quantity that must be determined is the shortest path for it to leave the sample. In the case of a cylindrical target, there are three separate surfaces the neutron could escape:
\begin{enumerate}
    \item The front surface at $z=0$,
    \item The rear surface at $z=n$, and
    \item The cylindrical surface at $x^2 + y^2 = R^2$.
\end{enumerate}
Therefore, there are three distances that must be calculated:
\begin{align*}
    d_{front}   &\equiv \text{Distance neutron must travel to intersect with front face} \\
    d_{back}    &\equiv \text{Distance neutron must travel to intersect with back face} \\
    d_{cyl}     &\equiv \text{Distance neutron must travel to intersect with cylindrical surface}
\end{align*}
The $d_{front}$ and $d_{back}$ terms are solved with
\begin{align}
    d_{front} &= \frac{z^k}{\Omega_{z}^{k}} \\
    d_{back} &= \frac{n - z^{k}}{\Omega^{k}_{z}}
\end{align}
while $d_{cyl}$ is is determined as
\begin{equation}
    \label{eq:cyl-intersection-distance}
    d_{cyl} = \frac{\sqrt{b^2 + ac} - b}{a}
\end{equation}
where
\begin{align}
    a &= \left( \Omega_{u}^{k} \right)^2 + \left( \Omega_{v}^{k} \right)^2\\
    b &= x^k \Omega_{u}^{k} + y^k\Omega_{v}^{k} \\
    c &= R^2 - \left( x^k \right)^2 - \left( y^k \right)^2
\end{align}

Once the $d_{front}, d_{back}, \text{ and } d_{cyl}$ distances are determined, the actual path length until escape is
\begin{equation}
    d^k =
    \begin{cases}
        \min{ \left( d_{cyl}, d_{front} \right)},   \qquad &\Omega_{w}^k < 0,\\
        d_{cyl},                                    \qquad &\Omega_{w}^k = 0, \\
        \min{ \left( d_{cyl}, d_{back} \right)},    \qquad &\Omega_{w}^k > 0,\\
    \end{cases}
\end{equation}

Next, a new set of total, capture, and elastic cross sections are sampled at energy $E^k$, which are then used to determine the sampled distance until the next collision,
\begin{equation}
    \label{eq:free-path-sampling}
    s^k = -\frac{1}{\Sigma_{t}} \ln{\left\{
        1 - \zeta \left[ 1 - \exp{\left( -\Sigma_t d^k \right)} \right]
    \right\}}
\end{equation}
where $\zeta$ is a randomly selected value between 0 and 1. This distance term $s^{k}$ is then used to calculate the location of the next scattering event,
\begin{equation}
    \label{eq:sampling-new-location}
    \overrightarrow{x}^{k+1} = \overrightarrow{x}^{k} + \Omega^{k}s^{k}
\end{equation}

\subsubsection{Sampling Energy}
\label{sec:sampling-energy-ms}
The energy in which the neutron leaves the scattering event, i.e., $E^{k+1}$, must be determined. This introduces its own complication, as $E^{k+1}$ is dependent on the mass of the isotope it interacts with, along with its scattering angle. 

First, addressing the mass dependency. This quantity must be sampled according to each isotopes calculated scattering cross section and their given abundances, as
\begin{equation}
    w_{j} = \frac{\delta_j \sigma_{j,el}^{k} }{\Sigma_{el}^k}
\end{equation}
This quantity is then used to calculate a cumulative weight, such that
\begin{equation}
    W_j = \sum_{j} w_j
\end{equation}
This defines the window of probability for the scattering event occurring on isotope $j$ as being between $(W_{j-1}, W_{j}]$.
A random number $\zeta$ is sampled between 0 and 1, is used such that
\begin{equation}
    W_{j-1} < \zeta \leq W_{j}
\end{equation}
determines that the neutron will be sampled as scattering off a nucleus with the mass $A_j$.

Next, addressing the angle dependency. The scattering angle is assumed to be isotropic, therefore can scatter with the angle $\phi$, determined as
\begin{equation}
    \phi^k = 2\pi\zeta
\end{equation}
Therefore, the energy $E^{k+1}$ is determined as
\begin{equation}
    E^{k+1} = E^{k} \frac{A_j^{2} + 2A_{j}\cos{\left(\phi^{k}\right) + 1}}{\left( A_{j} + 1 \right)^2}
\end{equation}

\subsubsection{Sampling Direction}
The final component of determining the new parameters of a scattering event include calculating the direction the neutron is traveling, $\overrightarrow{\Omega}^{k+1}$. Similar to \autoref{sec:sampling-energy-ms}, an angle $\theta^k$ must be sampled uniformly in the range $[0,2\pi]$. However $\theta$ must be converted from center of mass scale to lab scale, where
\begin{align}
    \cos{\theta'} &= \frac{1 + A_j \cos{\left( \theta^k \right)}}{\sqrt{1 + \left( A_j\right)^2 + 2A_j\cos{\left(\theta^k \right)}}} \\
    \sin{\theta'} &= \sqrt{1 - \left( \cos{\theta'} \right)^2}
\end{align}
while $\phi=\phi'$.
Determining $\overrightarrow{\Omega}^{k+1}$ from $\overrightarrow{\Omega}^{k}$, $\phi'$, and $\theta'$ proceed according to
\begin{equation}
    \label{eq:direction-calculation-ms}
    \overrightarrow{\Omega}^{k+1} = \begin{bmatrix}
        \Omega^{k+1}_u \\[8pt]
        \Omega^{k+1}_v \\[8pt]
        \Omega^{k+1}_w
    \end{bmatrix}
    = \begin{bmatrix}
        \frac{\Omega_{u}^{k} \Omega_{v}^{k}} { \sqrt{1 - \left(\Omega_{w}^{k}\right)^2 }} &
        \frac{-\Omega_v^k} { \sqrt{1 - \left(\Omega_{w}^{k}\right)^2 }} &
        \Omega_u \\[10pt]
        \frac{\Omega_{v}^{k} \Omega_{w}^{k}} { \sqrt{1 - \left(\Omega_{w}^{k}\right)^2 }} &
        \frac{-\Omega_u^k} { \sqrt{1 - \left(\Omega_{w}^{k}\right)^2 }} &
        \Omega_v \\[10pt]
        \frac{-1} { \sqrt{1 - \left(\Omega_{w}^{k}\right)^2 }} &
        0 &
        \Omega_w
    \end{bmatrix} \times
    \begin{bmatrix}
        \sin{\theta'}\cos{\phi'} \\[8pt]
        \sin{\theta'}\sin{\phi'} \\[8pt]
        \cos{\theta'}
    \end{bmatrix}
\end{equation}
However, in the case where $\left(\Omega_w^k\right)^2 = 1$, \autoref{eq:direction-calculation-ms} cannot be used as it would be diving by 0. Instead, the operation
\begin{equation}
        \overrightarrow{\Omega}^{k+1} = \begin{bmatrix}
        \Omega^{k+1}_u \\[8pt]
        \Omega^{k+1}_v \\[8pt]
        \Omega^{k+1}_w
    \end{bmatrix} = \Omega_w^k     \begin{bmatrix}
        \sin{\theta'}\cos{\phi'} \\[8pt]
        \sin{\theta'}\sin{\phi'} \\[8pt]
        \cos{\theta'}
    \end{bmatrix}
\end{equation}
is used instead. Finally, all required components to sample the subsequent scattering event are obtained: $\overrightarrow{x}^{k+1}$, $\overrightarrow{\Omega}^{k+1}$, and $E^{k+1}$.

\subsubsection{Killing neutrons}
\label{sec:killing-neutrons-ms}
A calculation must be made to ensure that sufficient scattering events are being accounted for, but not so

A weighted importance is used to determine the contribution of a particle to capture yield, and when to consider the neutron sufficiently unimportant. This is the same calculation being performed by default with MCNP\cite{mcnp}. 

\subsection{Calculating The Capture Correction Factor}

The term $w_{i,k}^{c}$ refers to the probability of a single neutron that is absorbed at the $k^{th}$ collision. 

The sequence of absorptions and scatters before the particle is killed can determine the yield of this particular neutron history,
\begin{equation}
    \label{eq:single-history-yield}
    Y_i = w_0^{c} + \sum_{k=1}^{K} w_{i,k}^{c}w_{i,k-1}^{s}
\end{equation}
Then taking the yield calculated for each neutron $Y_i$ over $N$ histories produces
\begin{equation}
    \label{eq:average-yield}
    \langle Y \rangle = \frac{1}{N}\sum{Y_i}^{N}
\end{equation}

Given this calculated average yield, $\langle Y \rangle$ and average capture cross section, $\langle \sigma_\gamma \rangle$, these are then used to calculate the capture correction factor,
\begin{equation}
    \label{eq:capture-correction}
    C_C = \frac{\langle Y \rangle}{n \langle \sigma_\gamma \rangle}
\end{equation}